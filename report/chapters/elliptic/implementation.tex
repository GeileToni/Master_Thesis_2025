%---Matrix-Vector System---------------------------------------------------------
\section{Matrix-Vector System}
\label{sec:matrix_vect_syst}
We will now in derive the fully discrete Matrix-Vector system given by
the variational form (\ref{eq:discrete_var_form_elliptic}). To do so let
$r \in \mathbb{N}$ denote the polynomial degree and consequently the element degree of freedom.
Note that in this thesis we will only consider global polynomial degrees, meaning one set polynomial degree for all elements.
Next let $\{\Phi_0,\ldots,\Phi_M\}$ be a basis of $V_h$, where $M = \dim(V_h)$.
We can represent the sought Galerkin approximation as $u_h = \sum_{j=0}^{M} \alpha_j \Phi_j\in V_h$ for coefficients
$\alpha_j \in \mathbb{R}$. Then (\ref{eq:discrete_var_form_elliptic}) is equivalent to:
\begin{equation*}
	\sum_{j=0}^{M} \alpha_j b_h(\Phi_j, \Phi_i) = \ell_h(\Phi_i) \qquad \forall i=0,\ldots,M
\end{equation*}
which corresponds to the system:
\begin{equation}
	\label{eq:fully_discrete_dg_system_elliptic}
	\textbf{Bu} = \textbf{l}
\end{equation}
for $ \textbf{B} \in \mathbb{R}^{M\times M}, [\textbf{B}]_{i,j} = b_h(\Phi_j, \Phi_i),
	\textbf{u} \in \mathbb{R}^M, [\textbf{u}]_j = \alpha_j,
	\textbf{l}\in\mathbb{R}^M, [\textbf{l}]_j = \ell_h(\Phi_j)$.

%---Basis-of-FE-Space------------------------------------------------------------
\section{Basis of Finite Element Space}
\label{sec:ell_basis}
There are many ways of choosing basis functions for finite element spaces. In this thesis we soley focus on elementwise nodal Lagrangian basis functions, although there are alternatives
which are just as valid, like the modal Legendre basis.
When choosing basis nodes for a lagrangian basis we have a couple of choices to make. \\ \\
1. \textit{Should the quadrature nodes coincide with the basis nodes?} \\
Having the quadrature nodes coincide with the basis nodes simplifies the calculations, because by the property of a Lagrangian basis the values of the basis functions at the quadrature nodes,
i.e the basis nodes coincide with the Kroneker delta. So a value matrix composed of each value at the quadrature nodes for each basis function (per element) is just
the identity matrix. Specifically we have: \\
Let $\xi_0,\ldots,\xi_r$ be the basis- and quadrature nodes and let $ \widehat{\phi}_0,\ldots,\widehat{\phi}_r$ be the Lagrangian basis, then 
\begin{equation*}
	\widehat{\phi}_i(\xi_j) = \delta_{i,j} := 
	\begin{cases}
		1	&,\text{ for }i=j \\
		0  &,\text{ for }i\neq j
	\end{cases}
\end{equation*}
therefore $\textbf{I} = \textbf{L}$ where $\textbf{I}$ is the identity and $ [\textbf{L}]_{i,j} = [\widehat{\phi}_i(\xi_j)] $ the basis shape-function value-matrix.
This reduces the computational cost, since there is no need to actually evaluate/interpolate the polynomials at the quadrature nodes. 
The clear downside is that we couple the exactness of the quadrature to the polynomial degree of the basis function. If for example we fix a polynomial degree $r \in \mathbb{N}$.
The Gauss-Lobatto quadrature with $r+1$ nodes is exact for polynomials of degree $2r-1$. When computing the mass matrix we have to calculate an 
integral of the form $ \int_I \Phi_j \Phi_i \text{d}x $ where $I \in \mathcal{T}_h$ is a element and $\Phi_i \vert_I, \Phi_j \vert_I \in \mathcal{P}^r(I)$ are basis functions.
This means $ \Phi_j \cdot \Phi_i \in  \mathcal{P}^{2r}$ and therefore we introduce an error. \\
% TODO: elaborate on error and remark that the convergence rate is not influenced after testing it

\noindent 2. \textit{Should there be basis nodes on the element boundaries?} \\
In contrast to continuous FEM we do not need to impose continuity over the element boundaries, so we have the option of allowing 
the nodes to be exclusively in the interior of the elements. This allows us to use Gauss-Legendre quadrature nodes in 1d for example, 
which is exact for polynomials of degree $2r+1$ for $r+1$ quadrature nodes. This would remedy the issue of approximating the mass matrix inexactly as described before when 
choosing the same quadrature nodes as basis nodes. The downside here is that we sill require boundary values to incorporate boundary conditions and calculating 
the penalty, consistency and symmetry terms (see matrices $\textbf{B}_{\text{cons}}, \textbf{B}_{\text{penal}} $ in section \ref{sec:stiff_assembly}), 
so the boundary values have to be interpolated. \\ 

In this thesis we will not try to justify the choice of basis function too much and use Gauss-Lobatto quadrature nodes as basis nodes.
This is a commonly used nodal basis in continuous 1d-FEM and we will also use it for our DG method. In Appendix A.2.3 of \cite{diPietro2012} the \textit{Fekete points}, which 
in 1d coincide with the Gauss-Lobatto points, are recommended 
with the argument that this nodal basis yields a mass matrix with an optimal condition number. 
For more detailed information on choosing basis functions and alternative approaches see for example
Appendix A.2 in \cite{diPietro2012}. \\ \\
Let $n\in \{1,\ldots,N\}$ and $I_n \in \mathcal{T}_h$ be an arbitrary element.
We denote $\hat{I} = (-1,1)$ the \textit{reference element} and $\displaystyle F_n : \hat{I} \to I_n, \, \xi \mapsto \frac{x_n + x_{n+1}}{2} + \frac{h_n}{2} \xi $
the \textit{element map}. This now allows us to define a basis on the reference element and
extend it to all elements using the element map. \\
For a fixed polynomial degree $r \geq 2$ let $\xi_0,\ldots,\xi_{r} \in [-1,1]$ be the
Gauss-Lobatto nodes. \\
\begin{equation*}
	\begin{tabular}{c|c}
		\centering
		$r = 2$ & $\{-1, 1\}$                                         \\
		$r = 3$ & $\{-1,0, 1\}$                                       \\
		$r = 4$ & $\{-1,-\frac{1}{\sqrt{5}}, \frac{1}{\sqrt{5}}, 1\}$ \\
		\vdots  & \vdots                                              \\
	\end{tabular}
\end{equation*}
The inner nodes are given by the roots of $L_{r-1}^{\prime}$, the derivative of the $r-1$-th Legendre polynomial.
We define the basis on the reference element as the Lagrangian nodal basis

\begin{equation}
	\label{def:lagrange_ref_basis}
	\widehat{\phi}_i(\xi) := \prod_{\substack{j = 0 \\ j\neq i}}^{r}\frac{\xi - \xi_j}{\xi_i - \xi_j},
	\qquad \text{for } i=0,\ldots,r
\end{equation}
and define the basis functions on the element $I_n$ as
\begin{equation}
	\phi^n_i : I_n \to \mathbb{R}, \quad \phi^n_i(x) := \widehat{\phi}_i(F_n^{-1}(x))
	\nonumber
\end{equation}
as a last step we extend the basis functions to the whole domain $\Omega$ by zero
\begin{equation}
	\Phi_i^n: \Omega \to \mathbb{R}, \quad \Phi_i^n(x) :=
	\begin{cases}
		\phi_i^n(x), & \text{for } x\in I_n \\
		0,           & \text{else}
	\end{cases}
	\label{def:lagrange_basis_of_Vh}
\end{equation}
for $n=0,\ldots,N$ and $i=0,\ldots,r$. Clearly we have $\text{span}( \widehat{\phi}_0, \ldots, \widehat{\phi}_r) = \mathcal{P}^r(\hat{I})$
and by extension $\text{span}(\Phi_0^0,\ldots,\Phi_r^N) = V_h^r(\mathcal{T}_h)$. It is essential
that our basis has only local support, meaning the basis functions are zero on most of the domain. This is the key property which
allows the final matrices to be sparse. Choosing basis functions with global support, the computational cost would be unfeasible for
small mesh sizes. \\
By having chosen a Lagrangian nodal basis the mesh nodes exactly coincide with the Gauss-Lobatto nodes on each
element.
To simplify the notation we introduce a \textit{local-to-global} index map
\begin{equation}
	\label{def:local_to_global_map}
	T: \{0,\ldots,N\} \times \{0,\ldots,r\} \to \{1,\ldots,M\}
\end{equation}
where $M = (r+1)(N+1) = \dim(V_h)$. $T$ takes an element index $n$ and a local basis function index $i$ as inputs
and returns the globally assigned node index $T(n,i)$. $T$ corresponds to the
\textit{connectivity matrix}. In the simplest case we have the global index ordered from left to right and get
$T(n,i) = nr + i$.

%---Matrix Assembly-------------------------------------------------------------------
\section{Stiffness Matrix Assembly}
\label{sec:stiff_assembly}
With the basis functions in (\ref{def:lagrange_basis_of_Vh}) defined we can now in detail investigate how
to assemble the matrix $\textbf{B}$ in (\ref{eq:fully_discrete_dg_system_elliptic}). To do so we firstly separate
the bilinear form $b_h$ into different components

\begin{align*}
	a_h(u,v)                & := \sum_{n=0}^N \int_{I_n} cu'v'\, \text{d}x                              \\
	b_h^{\text{cons}}(u,v)  & := \sum_{n=0}^{N+1} \{c(x_n)u'(x_n)\}[v(x_n)] + \{c(x_n)v'(x_n)\}[u(x_n)] \\
	b_h^{\text{penal}}(u,v) & := \sum_{n=0}^{N+1} \texttt{a}_n[u(x_n)][v(x_n)]                          \\
\end{align*}

Let $\displaystyle u_h = \sum_{m=0}^{N} \sum_{j=0}^{r} \alpha_j^m \Phi_j^m \in V_h$ denote
the Galerkin approximation, then as discussed in section \ref{sec:ell_basis} the discrete variational formulation (\ref{eq:discrete_var_form_elliptic}) is equivalent
to
\begin{equation}
	\sum_{m=0}^{N} \sum_{j=0}^{r} \alpha_j^m \Big(
	a_h(\Phi_j^m,\Phi_i^n) - b_h^{\text{cons}}(\Phi_j^m,\Phi_i^n) + b_h^{\text{penal}}(\Phi_j^m,\Phi_i^n)
	\Big)
	= \ell_h(\Phi_i^n), \qquad \forall n=0,\ldots,N, i=0,\ldots,r
\end{equation}
which corresponds to the matrix vector system (\ref{eq:fully_discrete_dg_system_elliptic})
where we can write
\begin{equation*}
	\textbf{B} = \textbf{A} - \textbf{B}_{\text{cons}} + \textbf{B}_{\text{penal}}
\end{equation*}
we will assemble the three (symmetric) matrices separately.
\begin{align*}
	 & [\textbf{B}]_{T(n,i),T(m,j)} = b_h(\Phi_j^m, \Phi_i^n),
	 & [\textbf{A}]_{T(n,i),T(m,j)} = a_h(\Phi_j^m, \Phi_i^n)                                \\
	 & [\textbf{B}_{\text{cons}}]_{T(n,i),T(m,j)} = b_h^{\text{cons}} (\Phi_j^m, \Phi_i^n)
	 & [\textbf{B}_{\text{penal}}]_{T(n,i),T(m,j)} = b_h^{\text{penal}} (\Phi_j^m, \Phi_i^n)
\end{align*}
where $T$ is given by (\ref{def:local_to_global_map})

\subsection{Assembly of A}
$\textbf{A}$ is assembled similarly to the standard stiffness matrix in continuous finite element. The main difference is that there is no overlap in the elementwise contributions.
Each set of local (element) basis functions only contributes to the integrals over said element.
We can rewrite $\textbf{A} = \sum_{s=0}^{N} \textbf{A}^{(s)}$, where
$[\textbf{A}^{(s)}]_{T(n,i),T(m,j)} = \int_{I_s} c \,(\Phi_j^m)' (\Phi_i^n)' \, \text{d}x$.
Now since we have $\text{supp}(\Phi_i^n)\subset I_n$ the only non-zero entries of $\textbf{A}^{(s)} $ are the ones where both $n=m=s$. Pulling back the integral
to the reference element using the chain rule and the substitution $F_s^{-1}(x) = \xi$ we find
\begin{equation*}
	\int_{I_s} c(x) \,(\Phi_j^s)'(x) (\Phi_i^s)'(x) \, \text{d}x =
	\frac{2}{h_s} \int_{\hat{I}} c(F_s(\xi)) \, \widehat{\phi}_j^{\, \prime}(\xi) \widehat{\phi}_i^{\, \prime}(\xi) \, \text{d}\xi
\end{equation*}
This integral now only depends on the reference shape functions, the element length $h_s$
and the values of the coefficient $c$. Using the Gauss-Lobatto quadrature rule we can approximate the integral
only requiring the values of $c$ at the nodes, whilst having the values of $\phi, \phi'$ preloaded.
The total assembly of $\textbf{A}$ can therefore be achieved by calculating a local contribution matrix
$\widehat{\textbf{A}}^{(s)} \in \mathbb{R}^{(r+1)\times (r+1)}$ for each element $I_s$ and adding it into $\textbf{A}$.
\begin{example}
	For $c\equiv 1$ with $\mathcal{P}^1$-elements $(r=1)$ we have
	\begin{equation*}
		\widehat{\textbf{A}}^{(s)} = \frac{1}{h_s}
		\begin{bmatrix}
			1  & -1 \\
			-1 & 1
		\end{bmatrix}
	\end{equation*}
\end{example}

\subsection{Assembly of B consistency part}
\label{subsec:assembly_cons}
As before we rewrite $\textbf{B}_{\text{cons}} = \sum_{s=0}^{N+1} \textbf{B}_{\text{cons}}^{(s)} $, where

\begin{equation*}
	[\textbf{B}_{\text{cons}}^{(s)}]_{T(n,i),T(m,j)} = \{c(x_s) \Phi_j^{m \, \prime} (x_s)\}[\Phi_i^n(x_s)] + \{c(x_s)\Phi_i^{n\, \prime} (x_s)\}[\Phi_j^m(x_s)]
\end{equation*}

\subsubsection{Interior Faces}
First let $s \in \{1,\ldots,N\}$ denote an interior face, we observe again, that the entries of
$\textbf{B}_{\text{cons}}^{(s)}$ can only be non-zero for an index tuple $(T(n,i),T(m,j))$ if
$n,m \in \{s, s-1\}$ by the local support of the basis functions.
Therefore assembling $ \textbf{B}_{\text{cons}}$ again comes down to calculating a local contribution matrix

\begin{equation*}
	\widehat{\textbf{B}}_{\text{cons}}^{(s)} =
	\begin{bmatrix}
		\textbf{C}_{\text{cons}}^{(s-1,s-1)} & \textbf{C}_{\text{cons}}^{(s-1,s)} \\
		\textbf{C}_{\text{cons}}^{(s,s-1)}   & \textbf{C}_{\text{cons}}^{(s,s)}
	\end{bmatrix}
	\in \mathbb{R}^{2(r+1) \times 2(r+1)}
\end{equation*}
for each interior face $x_s$ consisting of four blocks we will now lay out in more detail.
We discuss the boundary case separately.
Using again the local support of the basis functions we find
\begin{align*}
	 & [\textbf{C}_{\text{cons}}^{(s-1,s-1)}]_{i,j} = \frac{c(x_s^-)}{2} \Phi_j^{s-1\, \prime}(x_s^-) \Phi_i^{s-1}(x_s^-) + \frac{c(x_s^-)}{2} \Phi_i^{s-1\, \prime}(x_s^-) \Phi_j^{s-1}(x_s^-) \\
	 & [\textbf{C}_{\text{cons}}^{(s-1,s)}]_{i,j} = \frac{c(x_s^+)}{2} \Phi_j^{s\, \prime}(x_s^+) \Phi_i^{s-1}(x_s^-) - \frac{c(x_s^-)}{2} \Phi_i^{s-1\, \prime}(x_s^-) \Phi_j^{s}(x_s^+)       \\
	 & [\textbf{C}_{\text{cons}}^{(s,s)}]_{i,j} = -\frac{c(x_s^+)}{2} \Phi_j^{s\, \prime}(x_s^+) \Phi_i^{s}(x_s^+) - \frac{c(x_s^+)}{2} \Phi_i^{s\, \prime}(x_s^+) \Phi_j^{s}(x_s^+)
\end{align*}
where we have used the definitions of jump and average in (\ref{def:jump_average}).
Note that $ \textbf{C}_{\text{cons}}^{(s-1,s)} = (\textbf{C}_{\text{cons}}^{(s,s-1)})^T$ by the symmetry of the bilinear form $b_h^{\text{cons}}$.
Next we represent the values of the basis functions $ \Phi $ at the element boundary by the values of the reference
shape functions $\widehat{\phi}$
\begin{align}
	 & \Phi_i^s(x_s^+) = \widehat{\phi}_i(-1) ,                                                   & \Phi^{s-1}_i(x_s^-) = \widehat{\phi}_i(1) \label{eq:shape_fun_and_basis_fun_values_equality} \\
	 & \Phi_i^{s\, \prime }(x_s^+) = \frac{2}{h_{s}} \widehat{\phi}_i^{\,\prime}(-1),
	 & \Phi_i^{s-1\, \prime }(x_s^-) = \frac{2}{h_{s-1}} \widehat{\phi}_i^{\,\prime}(1) \nonumber
\end{align}
which finally yields
\begin{align*}
	 & [\textbf{C}_{\text{cons}}^{(s-1,s-1)}]_{i,j} = \frac{c(x_s^-)}{h_{s-1}} \widehat{\phi}_j^{\,\prime}(1) \widehat{\phi}_i(1) + \frac{c(x_s^-)}{h_{s-1}} \widehat{\phi}_i^{\,\prime} (1) \widehat{\phi}_j(1)  \\
	 & [\textbf{C}_{\text{cons}}^{(s-1,s)}]_{i,j} = \frac{c(x_s^+)}{h_{s}} \widehat{\phi}_j^{\,\prime} (-1) \widehat{\phi}_i (1) - \frac{c(x_s^-)}{h_{s-1}} \widehat{\phi}_i^{\,\prime} (1) \widehat{\phi}_j (-1) \\
	 & [\textbf{C}_{\text{cons}}^{(s,s)}]_{i,j} = -\frac{c(x_s^+)}{h_s} \widehat{\phi}_j^{\,\prime} (-1) \widehat{\phi}_i (-1) - \frac{c(x_s^+)}{h_s} \widehat{\phi}_i^{\,\prime} (-1) \widehat{\phi}_j (-1)
\end{align*}
\begin{example}
	Consider $c\equiv 1$ for $\mathcal{P}^1$-elements $(r=1)$
	with an equidistant mesh with meshsize $h$ we have
	\begin{equation*}
		\widehat{\textbf{B}}_{\text{cons}}^{(s)} = \frac{1}{h}
		\begin{bmatrix}
			0    & -1/2 & 1/2  & 0    \\
			-1/2 & 1    & -1   & 1/2  \\
			1/2  & -1   & 1    & -1/2 \\
			0    & 1/2  & -1/2 & 0
		\end{bmatrix}
	\end{equation*}
\end{example}

\subsubsection{Boundary Faces}
For $s\in \{0, N+1 \}$ and $x_s$ a boundary face we now have a smaller local contribution matrix
since the contribution can only come from the one element to which $x_s$ belongs. Meaning we have
$\widehat{\textbf{B}}_{\text{cons}}^{(0)}, \widehat{\textbf{B}}_{\text{cons}}^{(N+1)} \in \mathbb{R}^{(r+1)\times (r+1)}$
with
\begin{align*}
	 & [\widehat{\textbf{B}}_{\text{cons}}^{(0)}]_{i,j} =
	-\frac{2 c(x_0^+)}{h_{0}} \widehat{\phi}_j^{\,\prime}(-1) \widehat{\phi}_i(-1) - \frac{2 c(x_0^+)}{h_{0}} \widehat{\phi}_i^{\,\prime} (-1) \widehat{\phi}_j(-1)         \\
	 & [\widehat{\textbf{B}}_{\text{cons}}^{(N+1)}]_{i,j} =
	\frac{2 c(x_{N+1}^-)}{h_{N+1}} \widehat{\phi}_j^{\,\prime}(1) \widehat{\phi}_i(1) + \frac{ 2 c(x_{N+1}^-)}{h_{N+1}} \widehat{\phi}_i^{\,\prime} (1) \widehat{\phi}_j(1) \\
\end{align*}

\begin{example}
	For $c\equiv 1$ with $\mathcal{P}^1$-elements $(r=1)$ we have
	\begin{equation*}
		\widehat{\textbf{B}}_{\text{cons}}^{(0)} = \frac{1}{h_0}
		\begin{bmatrix}
			2  & -1 \\
			-1 & 0
		\end{bmatrix}
		,\qquad
		\widehat{\textbf{B}}_{\text{cons}}^{(N+1)} = \frac{1}{h_{N+1}}
		\begin{bmatrix}
			0  & -1 \\
			-1 & 2
		\end{bmatrix}
	\end{equation*}
\end{example}

\subsection{Assembly of B penalty part}
Again we rewrite $\textbf{B}_{\text{penal}} = \sum_{s=0}^{N+1} \textbf{B}_{\text{penal}}^{(s)}$, where
\begin{equation*}
	[\textbf{B}_{\text{penal}}^{(s)}]_{T(n,i), T(m,j)} = { \texttt{a}_s} [\Phi_j^m(x_s)] [\Phi_i^n(x_s)]
\end{equation*}
We proceed analogously to \ref{subsec:assembly_cons}.
\subsubsection{Interior Faces}
Let $s\in \{1,\ldots,N\} $ and $x_s$ denote an interior face. As before we have that
the entries of $ \textbf{B}_{\text{penal}}^{(s)} $ can only be nonzero for an index tuple
$ (T(n,i), T(m,j)) $ if $ n,m \in \{s, s-1\} $, similar to \ref{subsec:assembly_cons} we find
that the assembly boils down to adding up local contributions represented in a local contribution
matrix

\begin{equation*}
	\widehat{\textbf{B}}_{\text{penal}}^{(s)} =
	\begin{bmatrix}
		\textbf{C}_{\text{penal}}^{(s-1,s-1)} & \textbf{C}_{\text{penal}}^{(s-1,s)} \\
		\textbf{C}_{\text{penal}}^{(s,s-1)}   & \textbf{C}_{\text{penal}}^{(s,s)}
	\end{bmatrix}
	\in \mathbb{R}^{2(r+1) \times 2(r+1)}
\end{equation*}

and using (\ref{eq:shape_fun_and_basis_fun_values_equality}), and the definition of the penalization
parameter (\ref{def:penalization_function}) we specifically find

\begin{align*}
	 & [\textbf{C}_{\text{penal}}^{(s-1,s-1)}]_{i,j} = \texttt{a}_s \widehat{\phi}_j(1) \widehat{\phi}_i(1) \\
	 & [\textbf{C}_{\text{penal}}^{(s-1,s)}]_{i,j} = -\texttt{a}_s \widehat{\phi}_j(-1) \widehat{\phi}_i(1) \\
	 & [\textbf{C}_{\text{penal}}^{(s,s)}]_{i,j} = \texttt{a}_s \widehat{\phi}_j(-1) \widehat{\phi}_i(-1)
\end{align*}

where again by symmetry of the penalty term we have
$ \textbf{C}_{\text{penal}}^{(s-1,s)} = (\textbf{C}_{\text{penal}}^{(s,s-1)})^T$
and $\texttt{a}_s$ only depends on the two adjacent elements $I_{s-1}, I_s$.

\begin{example}
	Consider $c\equiv 1$ for $\mathcal{P}^1$-elements $(r=1)$
	with an equidistant mesh with meshsize $h$ we have
	\begin{equation*}
		\widehat{\textbf{B}}_{\text{penal}}^{(s)} = \frac{\sigma}{h}
		\begin{bmatrix}
			0 & 0  & 0  & 0 \\
			0 & 1  & -1 & 0 \\
			0 & -1 & 1  & 0 \\
			0 & 0  & 0  & 0
		\end{bmatrix}
	\end{equation*}
\end{example}

\subsubsection{Boundary Faces}
For $s \in  \{0, N+1\} $ we again have only the respective boundary element contributing.
So the local contribution matrices $ \widehat{\textbf{B}}_{\text{penal}}^{(s)} \in \mathbb{R}^{(r+1) \times (r+1)} $ satisfy

\begin{align*}
	[\widehat{\textbf{B}}_{\text{penal}}^{(0)}]_{i,j} =
	\texttt{a}_0 \widehat{\phi}_j(-1) \widehat{\phi}_i(-1), \qquad
	[\widehat{\textbf{B}}_{\text{penal}}^{(N+1)}]_{i,j} =
	\texttt{a}_{N+1} \widehat{\phi}_j(1) \widehat{\phi}_i(1) \\
\end{align*}

\begin{example}
	For $c\equiv 1$ with $\mathcal{P}^1$-elements $(r=1)$ we have
	\begin{equation*}
		\widehat{\textbf{B}}_{\text{penal}}^{(0)} = \frac{\sigma}{h_0}
		\begin{bmatrix}
			1 & 0 \\
			0 & 0
		\end{bmatrix}
		,\qquad
		\widehat{\textbf{B}}_{\text{penal}}^{(N+1)} = \frac{\sigma}{h_{N+1}}
		\begin{bmatrix}
			0 & 0 \\
			0 & 1
		\end{bmatrix}
	\end{equation*}
\end{example}

\subsection{System Vector Assembly}
We divide assembling the vector \textbf{l} in (\ref{eq:fully_discrete_dg_system_elliptic}) into two parts.
\begin{equation*}
	\textbf{l} = \textbf{l}_{\text{load}} + \textbf{l}_{\text{bc}}
\end{equation*}
First we recall the assembly of the load vector $\textbf{l}_{\text{load}}$, i.e. the vector containing the contributions of the forcing term $f$
and secondly we will describe how to add the Dirichlet boundary condition contributions ($\textbf{l}_{\text{bc}}$).
\subsubsection{Load Vector}
The assembly of the load vector is completely analogous to the continuous finite element case.
Using the local support of $\Phi_i^n$ we can rewrite

\begin{equation*}
	\int_{\Omega} f \Phi_i^n \text{d}x = \sum_{s=0}^{N} \int_{I_s} f \Phi_i^n \text{d} x = \int_{I_n} f(x) \Phi_i^n(x) \text{d} x
	= \frac{h_n}{2} \int_{-1}^{1} f(F_n(\xi))\widehat{\phi}_i(\xi)\text{d}\xi
\end{equation*}

meaning as before we can assemble $\textbf{l}_{\text{load}} = \sum_{s=0}^{N} \textbf{l}_{\text{load}}^{(s)}$ where
\begin{equation*}
    [\textbf{l}_{\text{load}}^{(s)}]_{T(n,i)} = \int_{I_s} f \Phi_i^n \text{d} x = \delta_{n,s} \frac{h_n}{2} \int_{-1}^{1} f(F_n(\xi))\widehat{\phi}_i(\xi)\text{d}\xi
\end{equation*}
which can be characterized by the local contribution vector $\widehat{\textbf{l}}_{\text{load}}^{\,(s)} \in \mathbb{R}^{r+1}$ defined as

\begin{equation*}
	[\widehat{\textbf{l}}_{\text{load}}^{\,(s)}]_{i}
	= \frac{h_s}{2} \int_{-1}^{1} f(F_s(\xi))\widehat{\phi}_i(\xi)\text{d}\xi
\end{equation*}
In practice we approximate the integral using the Gauss-Lobatto quadrature rule.

\begin{example}
	For $f$ piecewise constant (i.e. $f|_{I_s} \equiv f_s \in \mathbb{R} \quad \forall s=0,\ldots,N$) with $\mathcal{P}^1$-elements $(r=1)$ we have
	\begin{equation*}
		\normalfont \widehat{\textbf{l}}_{\text{load}}^{\,(s)} = \frac{f_s h_s}{2}
		\begin{bmatrix}
			1 \\
			1
		\end{bmatrix}
	\end{equation*}
\end{example}

\subsubsection{Dirichlet Boundary Condition Vector}
We have

\begin{equation*}
	[\textbf{l}_{\text{bc}}]_{T(n,i)} =
	-g_1c(x_{N+1}^-)\Phi_i^n(x_{N+1}^-) + g_0c(x_0^+)\Phi_i^n(x_0^+)
	+ \texttt{a}_{N+1}g_1\Phi_i^n(x_{N+1}^-) + \texttt{a}_0 g_0\Phi_i^n(x_{0}^+)
\end{equation*}

where the entries can clearly only be non-zero at indices corresponding to
boundary elements.
We characterize the assembly using the local contribution vectors
$\widehat{\textbf{l}}_{\text{bc}}^{\,(N+1)}, \widehat{\textbf{l}}_{\text{bc}}^{\,(0)} \in \mathbb{R}^{r+1}$
where

\begin{equation*}
	[\widehat{\textbf{l}}_{\text{bc}}^{\,(0)}]_{i} = g_0c(x_{0}^-) \widehat{\phi}_i(-1) + \texttt{a}_{0}g_0\widehat{\phi}_i(-1), \qquad
	[\widehat{\textbf{l}}_{\text{bc}}^{\,(N+1)}]_{i} = -g_1c(x_{N+1}^-) \widehat{\phi}_i(1) + \texttt{a}_{N+1}g_1\widehat{\phi}_i(1)
\end{equation*}