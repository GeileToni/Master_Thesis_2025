First we will consider a time-independent elliptic problem. The derivation
of the bilinear form and the final fully discrete scheme is mostly based on
Chapter 1 in \cite{riviere2008}, with cross references from other sources 
in the bibliography.
\section{Problem}
We consider the following elliptic model problem:
\begin{equation}
    -(c(x)u'(x))' = f(x) \qquad \forall x\in \Omega
\end{equation} 
\begin{equation}
    u(0) = g_0, u(1) = g_1
\end{equation}
Where $\Omega = (0,1)$ is the domain, $g_0, g_1 \in \mathbb{R}$ yield
Dirichlet boundary conditions, $f \in L^2(\Omega)$ and $c:\Omega \to \mathbb{R}$
satisfies:
\[
    c_{\min} \leq c(x) \leq c_{\max} \qquad \forall x\in \Omega
\]
for $0 < c_{\min} \leq c_{\max}$.
By multiplying by a test function and integrating over $\Omega$ we get the 
weak formulation: \\
Find $u \in H^1(\Omega)$ such that:
\begin{equation}
    a(u,v) = (f,v)_{L^2(\Omega)} \qquad \forall v \in C_c^{\infty}(\Omega)
\end{equation}
Where 
\[
    a:H^1(\Omega) \times H^1(\Omega) \to \mathbb{R}, \qquad (u,v) \mapsto \int_{\Omega} c(x)u'(x)v'(x) \text{d}x
\]  
defines the standard elliptic bilinear form on $H^1(\Omega)$ and $(u,v)_{L^2(\Omega)} = \int_{\Omega} uv \,\text{d}x$
denotes the $L^2$-inner product.

\section{Discretization}
Let $0=x_0 < ... < x_{N+1} = 1$, let $I_n = (x_n, x_{n+1})$ for $n = 0,..,N$ be the elements and $\mathcal{T}_h = \{I_n\}_{n=0}^N$ the partition
of $\Omega$.
We denote the element length by $h_n = x_{n+1} - x_{n}$ for $n=0,..N$ and the global meshsize by
$h = \max_{n} h_n$.
Next we define the discontinuous finite element space
\begin{equation}
    V_h^r(\mathcal{T}_h) = \{v \in L^2(\Omega) |\, v|_{I_n} \in \mathcal{P}^r(I_n) \} 
\end{equation}
where $\mathcal{P}^r(I_n)$ denotes the space of polynomials $p:I_n \to \mathbb{R}$ with $\deg(p) = r$
for $r \in \mathbb{N}$. When the context allows it, we will denote the
finite element space with just $V_h$ for simplicity. 
$V_h$ is our final approximation space in which the numerical solution
lays.
We observe that in contrast to a continuous finite element approximation space 
here the resulting solution is a priori discontinuous by construction 
of the space. To proceed we will require the following trace operators:

\begin{definition} 
    Let $v:\Omega \to \mathbb{R}$ be piecewise continuous and let $n \in
    \{1,..,N\}$
    \begin{enumerate}[label=\textnormal{(\roman*)}]
        \item We denote $v(x_n^+) := \lim_{x \to x_n^+} v(x), v(x_n^-) := \lim_{x \to x_n^-} v(x)$
        the limit from above/below.
        \item We write
        \[
            [v(x_n)]:= v(x_n^+) - v(x_n^-)
        \]
        for the \textbf{jump}, and
        \[
            \{v(x_n)\}:= \frac{v(x_n^+) + v(x_n^-)}{2}
        \]
        for the \textbf{average}.
    \end{enumerate}
\end{definition}



