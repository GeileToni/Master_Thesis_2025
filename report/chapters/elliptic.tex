First we will consider a time-independent elliptic problem. The derivation
of the bilinear form and the final fully discrete scheme is mostly based on
Chapter 1 in \cite{riviere2008}, with cross references from other sources 
in the bibliography.

%---Problem---------------------------------------------------------------
\section{Problem}
We consider the following elliptic model problem:
\begin{equation}
    \label{eq:elliptic_pde}
    -(c(x)u'(x))' = f(x) \qquad \forall x\in \Omega
\end{equation} 
\begin{equation}
    \label{eq:elliptic_pde_bc}
    u(0) = g_0, u(1) = g_1
\end{equation}
Where $\Omega = (0,1)$ is the domain, $g_0, g_1 \in \mathbb{R}$ yield
Dirichlet boundary conditions, $f \in L^2(\Omega)$ and $c:\Omega \to \mathbb{R}$
satisfies:
\[
    c_{\min} \leq c(x) \leq c_{\max} \qquad \forall x\in \Omega
\]
for $0 < c_{\min} \leq c_{\max}$.
By multiplying by a test function and integrating over $\Omega$ we get the 
weak formulation: \\
Find $u \in H^1(\Omega)$ such that:
\begin{equation}
    a(u,v) = (f,v)_{L^2(\Omega)} \qquad \forall v \in C_c^{\infty}(\Omega)
\end{equation}
Where 
\[
    a:H^1(\Omega) \times H^1(\Omega) \to \mathbb{R}, \qquad (u,v) \mapsto \int_{\Omega} c(x)u'(x)v'(x) \text{d}x
\]  
defines the standard elliptic bilinear form on $H^1(\Omega)$ and 
\[
    (u,v)_{L^2(\Omega)} = \int_{\Omega} uv \,\text{d}x
\]

denotes the $L^2$-inner product.

\section{Discretization}
Let $0=x_0 < \cdots < x_{N+1} = 1$, let $I_n = (x_n, x_{n+1})$ for $n = 0,\ldots,N$ be the elements and $\mathcal{T}_h = \{I_n\}_{n=0}^N$ the partition
of $\Omega$.
We denote the element length by $h_n = x_{n+1} - x_{n}$ for $n=0,\ldots,N$ and the global meshsize by
$h = \max_{n} h_n$.
Next we define the discontinuous finite element space
\begin{equation}
    V_h^r(\mathcal{T}_h) = \{v \in L^2(\Omega) |\, v|_{I_n} \in \mathcal{P}^r(I_n) \} 
\end{equation}
where $\mathcal{P}^r(I_n)$ denotes the space of polynomials $p:I_n \to \mathbb{R}$ with $\deg(p) = r$
for $r \in \mathbb{N}$. When the context allows it, we will denote the
finite element space with just $V_h$ for simplicity. 
$V_h$ is our final approximation space in which the numerical solution
lays.
We observe that in contrast to a continuous finite element approximation space 
here the resulting solution is a priori discontinuous by construction 
of the space. Furthermore we have here $V_h \not\subset H^1(\Omega)$.
This is especially apparent in 1d due to the Sobolev embedding $H^1(\Omega) \subset C^0(\Omega)$. \\
To proceed we will require the following trace operators:

\begin{definition} 
    Let $v:\Omega \to \mathbb{R}$ be piecewise continuous and let $n \in
    \{1,\ldots,N\}$
    \begin{enumerate}[label=\textnormal{(\roman*)}]
        \item We denote $v(x_n^+) := \lim_{x \searrow x_n} v(x), v(x_n^-) := \lim_{x \nearrow x_n} v(x)$
        the limit from above/below.
        \item We define the \textbf{jump} at $x_n$ as
        \[
        [v(x_n)]:= v(x_n^-) - v(x_n^+)
        \]
        and the \textbf{average} at $x_n$ as
        \[
            \{v(x_n)\}:= \frac{v(x_n^+) + v(x_n^-)}{2}
        \]
        furthermore by convention we set: 
        \[
            [v(x_0)] := -v(x_0^+),\quad [v(x_{N+1})] := v(x_{N+1}^-),\quad 
            \{v(x_0)\}:=v(x_0^+),\quad \{v(x_{N+1})\}:= v(x_{N+1}^-)
        \]  
    \end{enumerate}
\end{definition}

To derive the SIPDG variational formulation, let $v \in V_h$ be a test
function. For simplicity suppose for now that the coefficient $c \in C^1(\Omega)$ and
the exact solution $u \in H^2(\Omega) \subset C^1(\Omega)$. 
Due to the discontinuity of the test function in contrast to 
continuous FEM we multiply $u$ with $v$ on each element $I_n$
and integrate by parts locally
\begin{equation*}
    \int_{x_n}^{x_{n+1}} fv\, \text{d}x = -\int_{x_n}^{x_{n+1}} (cu')'\, \text{d}x 
    = \int_{x_n}^{x_{n+1}} cu'v'\, \text{d}x 
    - \left. cu'v\right|_{x_n}^{x_{n+1}} \qquad \forall n=0,\ldots,N
\end{equation*}
then sum over all elements
\begin{equation}
    \label{eq:elliptic_sipdg_var_form_incomplete}
    (f,v)_{L^2(\Omega)} = \sum_{n=0}^N \int_{I_n} cu'v'\, \text{d}x 
    -\sum_{n=0}^{N+1} [c(x_n)u'(x_n)v(x_n)]
\end{equation}
where we have used that $\sum_{n=0}^N \left. w \right|_{x_n}^{x_{n+1}} = w(x_{N+1}^-) - 
w(x_{N}^+) + w(x_{N}^-) - \cdots - w(x_1^+) + w(x_1^-) - w(x_0^+) = \sum_{n=0}^{N+1} [w(x_n)]$ for any piece-wise continuous function $w$.
\\
By our construction are $c, u'$ continuous on $\Omega$, this means 
\begin{equation}
    \label{eq:id_1_cu_jump_zero}
        [c(x_n)u'(x_n)v(x_n)] = c(x_n)u'(x_n)[v(x_n)] = \{c(x_n)u'(x_n)\}[v(x_n)] \qquad \forall n=0,\ldots,N+1
\end{equation}
and 
\begin{equation}
    \label{eq:id_2_u_jump_zero}
    [u(x_n)] = 0 \qquad \forall n=1,\ldots,N
\end{equation}
The next step is to symmetrize our currently non-symmetrical bilinear form.
For that subtract $\sum_{n=0}^{N+1} \{c(x_n)v'(x_n)\}[v(x_n)]$ on both sides of
(\ref{eq:elliptic_sipdg_var_form_incomplete}).

